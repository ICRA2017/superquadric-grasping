\section{superquadric-\/grasping}
\label{group__superquadric-grasping}\index{superquadric-\/grasping@{superquadric-\/grasping}}


Framework for grasping pose computation Version\-: 1.\-0.  


Framework for grasping pose computation Version\-: 1.\-0. \begin{DoxyAuthor}{Author}
Giulia Vezzani \href{mailto:giulia.vezzani@iit.it}{\tt giulia.\-vezzani@iit.\-it} \par
 
\end{DoxyAuthor}
\begin{DoxyCopyright}{Copyright}
Released under the terms of the G\-N\-U G\-P\-L v2.\-0 . 
\end{DoxyCopyright}
\hypertarget{group__superquadric-grasping_intro_sec}{}\subsection{Description}\label{group__superquadric-grasping_intro_sec}
The module computes a {\bfseries  pose reachable} by the robot hand and make it {\bfseries grasp and lift the object}. The complete pipeline is the following\-:


\begin{DoxyEnumerate}
\item Given the {\bfseries object model}, as a {\itshape superquadric}, and the {\bfseries  selected hand}, whose {\bfseries graspable volume} is represented by an {\itshape ellipsoid}, a reachable pose is computed.
\begin{DoxyItemize}
\item The {\bfseries object model} can be provided from a {\itshape configuration file} or by quering the \href{https://github.com/robotology/superquadric-model}{\tt superquadric-\/model}, that computes {\itshape online} a superquadric fitting the object of interest point cloud.
\item The user can select not only one hand, but also {\bfseries both}. In this case, two poses will be computed, one for the left and one for the right hand.
\end{DoxyItemize}
\item The computed pose/{\bfseries poses} is/are {\bfseries shown} together with the superquadric representing the volume graspable by the hand, overlapped to the superquadric modelling the object.
\item Then the user can {\bfseries ask the robot to grasp and lift the object}. If the poses for both of the arms are computed, the user can choose the best one.
\item The {\bfseries trajectory} is shown and the robot {\bfseries reaches} the desired pose.
\item The robot {\bfseries grasps} and {\bfseries lifts} the object.
\item Finally the hand goes back to the initial pose.
\end{DoxyEnumerate}\hypertarget{group__superquadric-grasping_parameters_sec}{}\subsection{Parameters}\label{group__superquadric-grasping_parameters_sec}

\begin{DoxyItemize}
\item --robot\-: Robot used in the test
\item --left\-\_\-or\-\_\-right\-: Hand used
\item --calib\-\_\-cam\-: Used calibrate points from vision or not
\item --lift\-: Lift the object after grasping or not
\item --dir\-: Appraching direction in hand frame
\item --eye\-: Eyes used to get image
\item --online\-: Work online or not
\item --calib\-\_\-cam\-: Number of points sampled on the hand ellipsoid
\item --distance\-: Distance used for approach
\item --distance1\-: Distance used for grasping
\item --distance\-: Name of the object for asking the superquadric to superquadric-\/model
\item --name\-File\-Out\-\_\-right\-: Name of output file for right
\item --name\-File\-Out\-\_\-left\-: Name of output file for left
\item --name\-File\-Solution\-\_\-right\-: Name of solution file for right
\item --name\-File\-Solution\-\_\-left\-: Name of solution file for left
\item --name\-File\-Trajectory\-: Name of trajectory file for selected hand
\item --tol\-: Tolerance for Ipopt solver
\item --constr\-\_\-viol\-\_\-tol\-: Tolerance of constraints in Ipopt solver
\item --acceptable\-\_\-iter\-: Acceptable iter in Ipopt solver
\item --max\-\_\-iter\-: Maximum number of iterations in Ipopt solver
\item --mu\-\_\-strategy\-: Mu strategy in Ipopt solver
\item --nlp\-\_\-scaling\-\_\-method\-: N\-L\-P scaling method in Ipopt solver 
\end{DoxyItemize}\hypertarget{group__superquadric-grasping_inputports_sec}{}\subsection{Input Ports}\label{group__superquadric-grasping_inputports_sec}

\begin{DoxyItemize}
\item /superquadric-\/grasping/img\-:i \mbox{[}yarp\-::sig\-::\-Image\-Of\-Pixel\-Rgb\mbox{]} \mbox{[}default carrier\-:tcp\mbox{]}\-: sends the image from the left camera
\item /superquadric-\/grasping/superq\-:rpc \mbox{[}yarp\-::sig\-::\-Bottle\mbox{]} \mbox{[}default carrier\-:rpc\mbox{]}\-: asks superquadric object model
\item /superquadric-\/grasping/camcalib\-:rpc \mbox{[}yarp\-::sig\-::\-Bottle\mbox{]} \mbox{[}default carrier\-:rpc\mbox{]}\-: asks calibrated points from vision
\end{DoxyItemize}\hypertarget{group__superquadric-grasping_outputports_sec}{}\subsection{Output Ports}\label{group__superquadric-grasping_outputports_sec}

\begin{DoxyItemize}
\item /superquadric-\/grasping/img\-:o \mbox{[}yarp\-::sig\-::\-Image\-Of\-Pixel\-Rgb\mbox{]} \mbox{[}default carrier\-:tcp\mbox{]}\-: shows the image from the left camera with the computed pose
\end{DoxyItemize}\hypertarget{group__superquadric-grasping_services_sec}{}\subsection{Services}\label{group__superquadric-grasping_services_sec}

\begin{DoxyItemize}
\item /superquadric-\/grasping/rpc \mbox{[}rpc-\/server\mbox{]}\-: service port . This service is described in superquadric\-Grasping\-\_\-\-I\-D\-L (idl.\-thrift) 
\end{DoxyItemize}